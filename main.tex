\documentclass[12pt]{beamer}
\usepackage[T2A]{fontenc}
\usepackage[utf8]{inputenc}
\usepackage[russian]{babel}
\usepackage{hyperref}

\title{1404. Легко взломать!}
\author{Затирахин Кирилл, Винников Кирилл}
%\date{Июнь 2022}

\usetheme{Berlin}

\begin{document}

\maketitle

\begin{frame}{Оглавление}
    \tableofcontents
\end{frame}

\section{Ссылки}

\begin{frame}{Ссылки}
     \href{https://github.com/ZatirahinKirill/practic--2022}{https://github.com/ZatirahinKirill/practic--2022}

     \href{https://github.com/VinnikovKirill/summer-2022}{https://github.com/VinnikovKirill/summer-2022}
\end{frame}

\section{Условие задачи}

\begin{frame}{Условие задачи}
    Дано зашифрованное слово длиной не более 100 символов. Необходимо получить изначальное сообщение, если известен алгоритм его шифрования.
    \begin{enumerate}
        \item Каждая буква заменяется соответствующим ей числом: a на 0, b на 1, ..., z на 25.
        \item К первому числу добавляется 5, ко второму числу добавляется первое число, к третьему — второе и т.д. 
        \item Если какое-то число превосходит 25, то оно заменяется остатком от деления на 26. 
        \item Полученные числа обратно заменяются буквами.
    \end{enumerate}
\end{frame}

\begin{frame}{Пример}

\setlength\tabcolsep{13PX}
\renewcommand*{\arraystretch}{2}

    \begin{table}
    \centering
    \begin{tabular}{|c|c|c|c|c|c|c|}
        \hline
        - & s & e & c & r & e & t\\
        \hline
        Шаг 1. & 18 & 4 & 2 & 17 & 4 & 19\\
        Шаг 2. & 23 & 27 & 29 & 46 & 50 & 69\\
        Шаг 3. & 23 & 1 & 3 & 20 & 24 & 17\\
        Шаг 4. & x & b & d & u & y & r\\
        \hline
    \end{tabular}
    \end{table}
    
\end{frame}

\section{Решение}

\begin{frame}{Решение}
\fontsize{15pt}{15pt}\selectfont
    \begin{equation}
        \begin{cases}
            \widetilde{n}_{1} = n_{1} + 5\\
            \widetilde{n}_{2} = n_{2} + \widetilde{n}_{1}\\
            \widetilde{n}_{3} = n_{3} + \widetilde{n}_{2}\\
            ...\\
            \widetilde{n}_{k} = n_{k} + \widetilde{n}_{k-1}\\
        \end{cases}
    \end{equation}
\end{frame}

\begin{frame}{Решение}
\fontsize{15pt}{15pt}\selectfont
    \begin{equation}
        \begin{cases}
            n_{1} = \widetilde{n}_{1} - 5\\
            n_{2} = \widetilde{n}_{2} - \widetilde{n}_{1} = \widetilde{n}_{2} - (n_{1} + 5)\\
            n_{3} = \widetilde{n}_{3} - \widetilde{n}_{2} = \widetilde{n}_{3} - (n_{1} + n_{2} + 5)\\
            ...\\
            n_{k} = \widetilde{n}_{k} - \widetilde{n}_{k-1} = \widetilde{n}_{k} - (\sum\limits_{i=1}^{k-1} n_{i} + 5)\\
        \end{cases}
    \end{equation}
\end{frame}

\begin{frame}{Решение}
\centering
\fontsize{15pt}{15pt}\selectfont
    \begin{equation}
        \begin{cases}
            0 \leq \widetilde{n}_{k} \leq 25\\
            \sum\limits_{i=1}^{k-1} n_{i} + 5 \geq 5\\
        \end{cases}
    \end{equation}
    Возможна ситуация, когда\\
    $\widetilde{n}_{k} < \sum\limits_{i=1}^{k-1} n_{i} + 5$
\end{frame}

%\begin{frame}{Решение}
%\centering
%\fontsize{15pt}{15pt}\selectfont
%    while (\widetilde{n}_{k} < \sum\limits_{i=1}^{k-1} n_{i} + 5)\\
%    \widetilde{n}_{k} = \widetilde{n}_{k} + 26\\
%\end{frame}

\end{document}
